\documentclass[12pt]{article}

\usepackage{amsmath,amssymb,amsthm}
\usepackage{enumitem}
\usepackage[margin=1in]{geometry}
\usepackage[colorlinks=true]{hyperref}
\usepackage{url}
\usepackage{xcolor}

\setlength{\parindent}{0pt}

\begin{document}

\title{CS 224W Lecture 1 Notes}
\author{Matthew Jin}
\date{September 26, 2023}
\maketitle

\section*{Types of Graphs}

A \textbf{heterogeneous graphs} $G$ is defined by a 4-tuple $(V, E, R, T)$ where nodes
$v_i \in V$ with node type $T(v_i)$ and edges $(v_i, r, v_j) \in E$ with
relation type $r \in R$. Each node and edge also have their own
attributes/annotations.

\smallskip
A \textbf{homogeneous graph} is a graph with a single node type and a single
relation type.

\smallskip
A \textbf{bipartite graph} is a graph whose nodes can be partitioned into two disjoint
sets $U$ and $V$ where every edge connects a node in $U$ to a node in $V$.

\section*{Types of Tasks}

\subsection*{Node level}

\textit{Predict some information about a single node in the graph.}

\medskip
The goal of \textbf{node-level classification} is to characterize the structure
and position of a node in the network. For example, in a semi-supervised
learning setting, the user may be given a partially labeled graph and be asked
to predict the labels of the unlabeled nodes. Properties such as node
importance, position, and the structure of its subgraph may be relevant
information for the classification task.

\medskip
A \textbf{graphlet} is a count vector of rooted subgraphs around a given
node. A graphlet is calculated from an alphabet of graph connection patterns.

% NOTE: a tikz plot would be helpful here

\subsection*{Edge level}

\textit{Predict some information about the links/relations between nodes.}

\medskip
The goal of \textbf{link prediction} is to predict new/missing links in a graph.
Links can be missing at random, or the task can be to predict the links at a
future time. For example, recommender systems require link predictions to
generate new recommendations, and link prediction in drug/protein interaction
graphs can be used to predict potential side effects.

\subsection*{Graph/Subgraph level}

\textit{Predict something about an entire graph or a subsection of it.}

\medskip
The goal of \textbf{graph-level prediction} is to predict some property of a
graph or subgraph given the graph structure. For example, Google Maps uses a
graph of locations (nodes) and roads (edges) to estimate the time a route takes.
As another example, graphs representing the structure of a molecule can be used
to predict the properties of the molecule.

\end{document}
